% Copyright 2013 David W. Hogg (NYU).  All rights reserved.

\documentclass{beamer}
\input{hogg_presentation} % hogg standard colors

\begin{document}
\whiteonblack

{\setbeamertemplate{background canvas}{%
\resizebox{\paperwidth}{!}{\includegraphics{%
abell1689.jpg}}}\begin{frame}[plain]~\end{frame}}

\begin{frame}
\begin{itemize}
\item the Universe is expanding
\item there are black holes
\item dark matter
\item dark energy
\item multiverse
\item \emph{how could we know these things?}
  \begin{itemize}
  \item \emph{simple} (compact) explanations of data
  \item \emph{quantitatively} explain \emph{multiple kinds} of data
  \item much more compact than the best \emph{alternatives}
  \end{itemize}
\end{itemize}
\end{frame}

{\setbeamertemplate{background canvas}{%
\resizebox{\paperwidth}{!}{\includegraphics{%
NGC_4298_UGC_7412_VCC_483_irg_clean.jpg}}}\begin{frame}[plain]~\end{frame}}

{\setbeamertemplate{background canvas}{%
\resizebox{\paperwidth}{!}{\includegraphics{%
NGC_5194_UGC_8493_VV_403_irg_clean.jpg}}}\begin{frame}[plain]~\end{frame}}

{\setbeamertemplate{background canvas}{%
\resizebox{\paperwidth}{!}{\includegraphics{%
NGC_7814_UGC_8_irg_clean.jpg}}}\begin{frame}[plain]~\end{frame}}

\begin{frame}
\frametitle{cold dark matter}
\begin{itemize}
\item model designed for \emph{simplicity} not realism
  \begin{itemize}
  \item cold
  \item non-interacting (or weak-scale)
  \item Gaussian, adiabatic initial fluctuations
  \end{itemize}
\item this model \emph{must be wrong!}
\end{itemize}
\end{frame}

\begin{frame}
\frametitle{precise tests of gravity}
\begin{itemize}
\item cosmic microwave background
\item large-scale structure (few-percent-level)
\item Solar System (13 decimal places of accuracy!)
\end{itemize}
\end{frame}

{\setbeamertemplate{background canvas}{%
\resizebox{!}{\paperheight}{\includegraphics{%
1920x1080_Planck11-full-size.jpg}}}\begin{frame}[plain]~\end{frame}}

{\setbeamertemplate{background canvas}{%
\resizebox{!}{\paperheight}{\includegraphics{%
sdss..jpg}}}\begin{frame}[plain]~\end{frame}}

\begin{frame}
\frametitle{astronomical scales}
\begin{itemize}
\item Universe --- $10^{26}$~m --- 14 billion yrs
\item Large-scale structure --- $10^{24}$~m
\item Galaxy --- $3\times 10^{20}$~m --- 0.2 billion yrs
\item Solar System --- $10^{12}$~m --- few yrs
\end{itemize}
\end{frame}

\begin{frame}
\frametitle{dark-matter options}
\begin{itemize}
\item WIMPs
\item axions or other particle exotica
\item tiny black holes
\item \emph{not} dust, rocks, molecules, comets, planets, or anything \emph{made of atoms}
\end{itemize}
\end{frame}

{\setbeamertemplate{background canvas}{%
\resizebox{!}{\paperheight}{\includegraphics{%
Abrams40.jpg}}}\begin{frame}[plain]~\end{frame}}

{\setbeamertemplate{background canvas}{%
\resizebox{\paperwidth}{!}{\includegraphics{%
md.jpg}}}\begin{frame}[plain]~\end{frame}}

\end{document}
