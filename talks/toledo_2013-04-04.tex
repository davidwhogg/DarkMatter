% Copyright 2013 David W. Hogg (NYU).  All rights reserved.

\documentclass{beamer}
\usepackage{amssymb,amsmath,mathrsfs}
\usepackage{listings}
\usecolortheme{default}

%%% color commands
\newcommand{\whiteonblack}{%
  \colorlet{fg}{white}
  \colorlet{bg}{black}
  \setbeamercolor{normal_text}{fg=white,bg=black}
  \setbeamercolor{background canvas}{fg=white,bg=black}
  \setbeamercolor{alerted_text}{fg=yellow}
  \setbeamercolor{example_text}{fg=white}
  \setbeamercolor{structure}{fg=white}
  \setbeamercolor{palette_quaternary}{fg=white}
}
\newcommand{\blackonwhite}{%
  \colorlet{fg}{black}
  \colorlet{bg}{white}
  \setbeamercolor{normal_text}{fg=black,bg=white}
  \setbeamercolor{background canvas}{fg=black,bg=white}
  \setbeamercolor{alerted_text}{fg=blue}
  \setbeamercolor{example_text}{fg=black}
  \setbeamercolor{structure}{fg=black}
  \setbeamercolor{palette_quaternary}{fg=black}
}
\xdefinecolor{pink}{rgb}{1.0,0.9,0.9}

%%% size and shape commands
\newlength{\figurewidth}
\setlength{\figurewidth}{0.9\textwidth}
\newlength{\figureheight}
\setlength{\figureheight}{0.9\textheight}

%%% text commands
\newcommand{\project}[1]{\textsl{#1}}
  \newcommand{\an}{\project{Astrometry.net}}
  \newcommand{\flickr}{\project{flickr}}
  \newcommand{\gaia}{\project{Gaia}}
  \newcommand{\galex}{\project{GALEX}}
  \newcommand{\GALEX}{\galex}
  \newcommand{\hst}{\project{HST}}
  \newcommand{\hipparcos}{\project{Hipparcos}}
  \newcommand{\lsst}{\project{LSST}}
  \newcommand{\sdss}{\project{SDSS}}
  \newcommand{\sdssiii}{\project{SDSS-III}}
  \newcommand{\boss}{\sdssiii\ \project{BOSS}}
  \newcommand{\osss}{\project{OSSS}}
  \newcommand{\vo}{\project{VO}}
  \newcommand{\rttd}{\project{Right Thing To Do}$^{\mbox{\scriptsize\sffamily{TM}}}$}
\newcommand{\foreign}[1]{\textit{#1}}
\newcommand{\latin}[1]{\foreign{#1}}
  \newcommand{\cf}{\latin{cf.}}
  \newcommand{\eg}{\latin{e.g.}}
  \newcommand{\etal}{\latin{et~al.}}
  \newcommand{\etc}{\latin{etc.}}
  \newcommand{\ie}{\latin{i.e.}}
  \newcommand{\vs}{\latin{vs.}}

%%% math-mode commands
\newcommand{\unit}[1]{\mathrm{#1}}
  \newcommand{\rad}{\unit{rad}}
  \newcommand{\s}{\unit{s}}
  \newcommand{\yr}{\unit{yr}}
  \newcommand{\km}{\unit{km}}
  \newcommand{\kmps}{\km\,\s^{-1}}
\newcommand{\mmatrix}[1]{\boldsymbol{#1}}
\newcommand{\tv}[1]{\boldsymbol{#1}}
\newcommand{\dd}{\mathrm{d}}
\newcommand{\given}{\,|\,}

%%% code commands
\lstset{language=Python,
        basicstyle=\tiny\ttfamily,
        showspaces=false,
        showstringspaces=false,
        tabsize=2,
        breaklines=false,
        breakatwhitespace=true,
        identifierstyle=\ttfamily,
        keywordstyle=\bfseries\color[rgb]{0.133,0.545,0.133},
        commentstyle=\color[rgb]{0.133,0.545,0.133},
        stringstyle=\color[rgb]{0.627,0.126,0.941},
        }
 % hogg standard colors

\begin{document}
\whiteonblack

{\setbeamertemplate{background canvas}{%
\resizebox{\paperwidth}{!}{\includegraphics{%
abell1689.jpg}}}\begin{frame}[plain]~\end{frame}}

\begin{frame}
\begin{itemize}
\item the Universe is expanding
\item there are black holes
\item dark matter
\item dark energy
\item multiverse
\item \emph{how could we know these things?}
  \begin{itemize}
  \item \emph{simple} (compact) explanations of data
  \item \emph{quantitatively} explain \emph{multiple kinds} of data
  \item much more compact than the best \emph{alternatives}
  \end{itemize}
\end{itemize}
\end{frame}

{\setbeamertemplate{background canvas}{%
\resizebox{\paperwidth}{!}{\includegraphics{%
NGC_4298_UGC_7412_VCC_483_irg_clean.jpg}}}\begin{frame}[plain]~\end{frame}}

{\setbeamertemplate{background canvas}{%
\resizebox{\paperwidth}{!}{\includegraphics{%
NGC_5194_UGC_8493_VV_403_irg_clean.jpg}}}\begin{frame}[plain]~\end{frame}}

{\setbeamertemplate{background canvas}{%
\resizebox{\paperwidth}{!}{\includegraphics{%
NGC_7814_UGC_8_irg_clean.jpg}}}\begin{frame}[plain]~\end{frame}}

\begin{frame}
\frametitle{cold dark matter}
\begin{itemize}
\item model designed for \emph{simplicity} not realism
  \begin{itemize}
  \item cold
  \item non-interacting (or weak-scale)
  \item Gaussian, adiabatic initial fluctuations
  \end{itemize}
\item this model \emph{must be wrong!}
\item it was made to explain galaxies and it ended up \emph{explaining the whole Universe}
\end{itemize}
\end{frame}

{\setbeamertemplate{background canvas}{%
\resizebox{!}{\paperheight}{\includegraphics{%
1920x1080_Planck11-full-size.jpg}}}\begin{frame}[plain]~\end{frame}}

{\setbeamertemplate{background canvas}{%
\resizebox{\paperwidth}{!}{\includegraphics{%
sdss.jpg}}}\begin{frame}[plain]~\end{frame}}

\begin{frame}
\frametitle{precise tests of gravity}
\begin{itemize}
\item cosmic microwave background
\item large-scale structure (few-percent-level)
\item Solar System (13 decimal places of accuracy!)
\item laboratory tests
\end{itemize}
\end{frame}

\begin{frame}
\frametitle{why not just modify gravity?}
\begin{itemize}
\item Universe --- $10^{26}$~m --- 14 billion yrs
\item large-scale structure --- $10^{24}$~m
\item Galaxy --- $3\times 10^{20}$~m --- 0.2 billion yrs
\item Solar System --- $10^{12}$~m --- few yrs
\item laboratory tests --- $10^{-4}$~m (seriously!)
\end{itemize}
\end{frame}

\begin{frame}
\frametitle{dark-matter options}
\begin{itemize}
\item weakly interacting massive particles (WIMPs)
  \begin{itemize}
  \item (the only strongly motivated model from particle theory)
  \end{itemize}
\item axions or other particle exotica
\item tiny black holes
\item \emph{not} dust, rocks, molecules, comets, planets, or anything \emph{made of atoms}
  \begin{itemize}
  \item (the cold-dark matter model \emph{also} sets the total density of atoms)
  \end{itemize}
\end{itemize}
\end{frame}

\begin{frame}
\frametitle{direct detection and annihilation}
\begin{itemize}
\item no fully convincing signals
\item some tantalizing hints
\item possibly game-changing, inexpensive science
\end{itemize}
\end{frame}

{\setbeamertemplate{background canvas}{%
\resizebox{!}{\paperheight}{\includegraphics{%
Abrams40.jpg}}}\begin{frame}[plain]~\end{frame}}

{\setbeamertemplate{background canvas}{%
\resizebox{\paperwidth}{!}{\includegraphics{%
md.jpg}}}\begin{frame}[plain]~\end{frame}}

\end{document}
