\documentclass[12pt]{article}
\begin{document}

\section{mass inference}

How can you infer the mass of a distant, massive stellar (or gaseous)
system?  You can measure positions (or some projection thereof) and
you can measure velocities (or some projection thereof) but you have
no chance in hell of measuring accelerations.  This situation is very
different from that in the Solar System, where inference proceeded not
just from acceleration measurements, but full orbit mapping.

The key idea is that in order to go from positions and velocities to
inferences about forces, you have to make \emph{strong assumptions},
like that we don't observe at a special time, or that the system under
study is long lived relative to its characteristic lengthscale (size)
divided by its characteristic velocity (velocity spread or
dispersion).

In case this makes you uncomfortable, reflect on the fact that
\emph{all} astronomical inferences depend on these kinds of
assumptions.  For example, we know the Universe is expanding, because
otherwise the observations have to put us in a very special place or
at a very special time.  We know the age of the Universe (in part)
because we can run the clock back to a singularity in the position
distribution.  We don't think the velocities and positions we see are
likely to show the relevant regularities \emph{by chance}; we think
that any typical observer (whatever that means) would observe the
same.

The assumptions are \emph{testable} of course; usually the assumptions
permit us to make predictions about other observed systems or permit
us to explain other regularities in the data.  The assumptions can be
dropped, but when they are dropped, the explanations become
\emph{implausible}.

There are other ways to measure mass, but not many!  Gravitational
lensing is brought up as a counter-example, but it isn't really; all
we measure is the celestial positions (angles) from which photons
arrive; gravitational lensing is inferred by assuming that the photons
aren't arranged into gravitational-lensing configurations by
chance---or that the gravitational lensing explanation is
\emph{more plausible} than the alternatives.

\section{astronomical scales}

Just as a reminder, the relevant length and time scales for astronomy
and cosmology are the following: The Universe is 14~billion years old,
and has a size measured in Gpc or multiples of $10^{26}$~m, depending
on your definition of ``size'' (which in cosmology is a bit slippery).
Galaxies are separated by Mpc scales ($\sim 10^{23}$~m), and have
taken the full age of the Universe to evolve to their current state.
This evolution has involved mass assembly---the Universe started in a
near-homogeneous state---and star formation and plenty of dynamical
churning.  A galaxy like the Milky Way lives in a dark-matter
concentration that is probably about $200$~kpc ($\sim 10^{22}$~m) in
radius.  The stars in the Galaxy extend out to about 20~kpc ($\sim
10^{21}$~m) and orbit the Galactic center at $\sim 200$~km\,s$^{-1}$,
making the ``dynamical time'' for the Milky Way something like a few
hundred million yr.  The Solar System is tiny on these scales; the
Earth-Sun distance is $\sim 10^{11}$~m and the furthest reaches of the
Solar System lie at 100 times that distance.  The characteristic
dynamical times of the Solar System are measured in years.

\section{why do we think there is dark matter?}

The origin of this discovery is with Zwicky (1930s), who found that if
you count up the stars you can see in massive galaxy clusters
(concentrated groups of Milky-Way-like galaxies) and then compare that
to their sizes and velocity dispersions, the galaxy clusters would not
conceivably be gravitationally bound objects.  That is, they would
either have to be bound together by forces \emph{other than gravity}
or else be \emph{chance superpositions of unrelated galaxies passing
  by} or else there must be \emph{mass that doesn't shine like stars}.
Zwicky (correctly) rejected the second option---pure chance---as
outrageously unlikely.  Interestingly, between the first and third,
his intuition said that it was more likely that there be unseen mass.
Unseen mass seemed natural (and still does to many of us); after all,
why should \emph{all} the stuff in the Universe bind into stars and
shine?  Even in the 1930s, it was known that parts of galaxies were
gaseous and ionized plasma, both of which were hard to observe; why
not other unseen components?

In the 1970s, Vera Rubin and collaborators showed that Zwicky's
paradox (relative velocities and sizes too large given the visible
stars) applied to generic galaxies also.  They found this by measuring
the velocities of gas (not stars) in the galaxies, but the velocities
of the stars and the gas are similar.  Galaxies are obviously---from
their stellar content and isolation one from another---very long-lived
objects, so the conclusions were strong: Either there are forces
beyond gravity involved, or else there is invisible mass.  It is in
this era, the unseen mass got named (possibly badly) ``dark matter''.



\end{document}
