\documentclass[12pt]{article}
\begin{document}

\section{mass inference}

How can you infer the mass of a distant, massive stellar (or gaseous)
system?  You can measure positions (or some projection thereof) and
you can measure velocities (or some projection thereof) but you have
no chance in hell of measuring accelerations.  This situation is very
different from that in the Solar System, where inference proceeded not
just from acceleration measurements, but full orbit mapping: Kepler
(cite) observed that the planets travel on elliptical orbits with the
Sun at one focus, and Hooke (?) and Newton (cite) showed that this is
a consequence of a particular acceleration law.

The key idea is that in order to go from positions and velocities at a
snapshot---a single moment in time, which is all we get on
cosmological scales---to inferences about forces (accelerations), you
have to make \emph{strong assumptions}, like that we don't observe at
a special time, or that the system under study is long lived relative
to its characteristic lengthscale (size) divided by its characteristic
velocity (velocity spread or dispersion).

In case this makes you uncomfortable, reflect on the fact that
\emph{all} astronomical inferences depend on these kinds of
assumptions.  For example, we know the Universe is expanding, because
otherwise the observations have to put us in a very special place or
at a very special time.  We know the age of the Universe (in part)
because we can run the clock back to a singularity in the position
distribution.  We don't think the velocities and positions we see are
likely to show the relevant regularities \emph{by chance}; we think
that any typical observer (whatever that means) would observe the
same.

The assumptions are \emph{testable} of course; usually the assumptions
permit us to make predictions about other observed systems or permit
us to explain other regularities in the data.  The assumptions can be
dropped, but when they are dropped, the explanations become
\emph{implausible}.

There are other ways to measure mass, but not many!  Gravitational
lensing is brought up as a counter-example, but it isn't really; all
we measure is the celestial positions (angles) from which photons
arrive; gravitational lensing is inferred by assuming that the photons
aren't arranged into gravitational-lensing configurations by
chance---or that the gravitational lensing explanation is \emph{more
  plausible} than the alternatives.  All that said, it is remarkable
and encouraging to our understanding of gravity that
gravitational-lensihg and stellar dynamical measurements of galaxy and
galaxy cluster masses appear to be very consistent (cite Bolton,
others?).

\section{astronomical scales}

Just as a reminder, the relevant length and time scales for astronomy
and cosmology are the following: The Universe is 14~billion years old
($??$~s), and has a size measured in Gpc or multiples of $10^{26}$~m,
depending on your definition of ``size'' (which in cosmology is a bit
slippery).  Galaxies are separated by Mpc scales ($\sim 10^{23}$~m),
and have taken the full age of the Universe to evolve to their current
state.  This evolution has involved mass assembly---the Universe
started in a near-homogeneous state---and star formation and plenty of
dynamical churning, all of which has taken billions of yr ($\sim
10^17$~s).  A galaxy like the Milky Way lives in a dark-matter
concentration that is probably about $200$~kpc ($\sim 10^{22}$~m) in
radius.  The stars in the Galaxy extend out to about 20~kpc ($\sim
10^{21}$~m) and orbit the Galactic center at $\sim 200$~km\,s$^{-1}$,
making the ``dynamical time'' for the Milky Way something like a few
hundred million yr.  The Solar System is tiny on these scales; the
Earth-Sun distance is $\sim 10^{11}$~m and the furthest reaches of the
Solar System lie at 100 times that distance.  The characteristic
dynamical times of the Solar System are measured in years ($\sim
10^8$~s).

...gravity is extremely well tested (that is, observations of
gravitational action are very precise) on some of these
scales... CMB... LSS... Galaxy-scale gravitational lensing... Solar
System...

\section{cold dark matter}

The origin of this discovery is with Zwicky (1930s), who found that if
you count up the stars you can see in massive galaxy clusters
(concentrated groups of Milky-Way-like galaxies) and then compare that
to their sizes and velocity dispersions, the galaxy clusters would not
conceivably be gravitationally bound objects.  That is, they would
either have to be bound together by forces \emph{other than gravity}
or else be \emph{chance superpositions of unrelated galaxies passing
  by} or else there must be \emph{mass that doesn't shine like stars}.
Zwicky (correctly) rejected the second option---pure chance---as
outrageously unlikely.  Interestingly, between the first and third,
his intuition said that it was more likely that there be unseen mass.
Unseen mass seemed natural (and still does to many of us); after all,
why should \emph{all} the stuff in the Universe bind into stars and
shine?  Even in the 1930s, it was known that parts of galaxies were
gaseous and ionized plasma, both of which were hard to observe; why
not other unseen components?

In the 1970s, Vera Rubin and collaborators showed that Zwicky's
paradox (relative velocities and sizes too large given the visible
stars) applied to generic galaxies also.  They found this by measuring
the velocities of gas (not stars) in the galaxies, but the velocities
of the stars and the gas are similar.  Galaxies are obviously---from
their stellar content and isolation one from another---very long-lived
objects, so the conclusions were strong: Either there are forces
beyond gravity involved, or else there is invisible mass.  It is in
this era, the unseen mass got named (possibly badly) ``dark matter''.

As a straw-man model, Jim Peebles and others in the 1980s came up with
the most absolutely simple model conceivable for the dark matter: That
it be (a)~non-interacting dust with no interactions other than
gravitational, that it be (b)~cold (have low velocity dispersion at
early times), that it (c)~be perturbed away from perfect homogeneity
by perfectly Gaussian fluctuations with some simple spectrum, that
these perturbations be (d)~adiabatic (temperature and density
fluctuations related as you would expect if you assumed constant
entropy), and (e)~possibly there is some cosmological constant.  These
assumptions were made not because they are particularly plausible, but
rather to make a straw-man model that is calculable simply with the
technology available in the 1980s.

This model turns out to be absolutely right, at least on large scales:
The fluctuations in the Cosmic Microwave Background and the
fluctuations in the galaxy population in the Universe have both been
measured more and more precisely every few years, starting in the late
1980s.  Every large-scale measurement of the cosmogonic model has
confirmed the adiabatic cold-dark-matter model (with cosmological
constant) for structure formation in the Universe.  Indeed this model
is so well tested now on length scales from, say, 10~Mpc to the Hubble
scale, that if it \emph{doesn't} turn out to be the right model, the
right model has to make predictions that pretty precisely
(percent-level) reduce to the predictions of the CDM model in the
large-scale limit.  That's a very strong constraint on dark-matter
theories.

And importantly, the CDM model was developed to explain galaxy
observations, and it does an even more precise job on cosmological
scales, explaining the growth of large-scale structure.  From an
epistemological perspective, this is important support for the theory.
For what follows, an important aspect of the CDM theory is that it
predicts that the dark matter associated with a typical galaxy
(including the Milky Way) will be far larger in extent (radius) than
the stellar content.  This relates, fundamentally, to the fact that
there are dissipative processes in the ``light sector'' of plasma and
atoms and molecules, but the dark matter (by assumption) is
non-interacting, except through gravity, which is a perfectly
conservative force.

\section{why not just modify gravity?}

Obviously, if you have problems with gravitational explanations of
condensed objects in the Universe, it could be because you don't know
the masses, \emph{or} it could be because you don't know \emph{the law
  of gravity}.  Perhaps gravitational forces (or accelerations) vary
like $1/r^2$ on Solar-System scales but have some other dependence or
strength on Galaxy and cosmological scales?

This is a great project; if gravity can be modified to remove the need
for dark matter, that would be a huge breakthrough.  Right now, things
look unlikely.  Many attempts have been made to modify gravity---make
changes to or replace Einstein's equation of General Relativity---and
none do the whole thing: Any modification to gravity has to reproduce
all the predictions (or explanations) that the CDM model makes so
successfully on the scale of the Universe as a whole, \emph{and}
reproduce the extremely precise measurements of gravity that have been
made in the Solar System.

The Solar System model is extremely precise now; planet positions are
currently predicted and measured at cm scales, over length scales of
$10^12$~m.  That is, there are strong constraints on any theory of
gravity, at small scales (Solar System) and large scales (CMB and
LSS); if you can't match both, then you aren't doing as well as
Einstein gravity plus CDM.

All that said, the project of modifying gravity is not over or dead.
Many people are working on this, and it is rational to do so: There
are many ways gravity might be modified, and only a small subset of
possible modifications has been tested.  Right now there are no
promising directions, but that doesn't mean that promising directions
won't be found in the near future.

\section{CDM on galaxy scales}

I mentioned above that the CDM model predicts a larger spatial extent
for the dark matter than the stellar component of galaxies, including
the Milky Way.  In general, the CDM model, taken literally---that is,
under strict assumptions of non-interacting, cold, adiabatic, and
Gaussian---makes extremely strong predictions for the spatial
distribution of dark matter around galaxies.

...virial extent.

...fractal-like substructure.

...not directly observable in the stellar distribution.

\section{annihilation and direct detection}

...annihilation doesn't seem to be happening.  It would be great.

...direct detection would be huge.  It doesn't tell you much of
cosmological interest, however.

\section{imaging gravitational orbits}

...can't see orbits in real time

...tidal disruption can reveal (indirectly) gravitational orbits.

...large collections of tidal streams could reveal non-trivial spatial
and temporal structure.

\end{document}
