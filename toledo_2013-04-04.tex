% Copyright 2013 David W. Hogg (NYU).  All rights reserved.

\documentclass{beamer}
\input{hogg_presentation} % hogg standard colors

\begin{document}

\begin{frame}
\begin{itemize}
\item Universe is expanding
\item black holes
\item dark matter
\item dark energy
\item multiverse
\item how \emph{could} we know these things?
  \begin{itemize}
  \item \emph{simple} (compact) explanations of data
  \item explain \emph{multiple kinds} of data
  \item much more compact than the alternatives
  \end{itemize}
\end{itemize}
\end{frame}

\begin{frame}
~[galaxy cluster]
\end{frame}

\begin{frame}
~[Milky-Way-like galaxy]
\end{frame}

\begin{frame}
~[same, zoomed out]
\end{frame}

\begin{frame}
\begin{itemize}
\item cold
\item non-interacting (or weak-scale)
\item Gaussian
\item adiabatic
\item (cosmological constant)
\item this model \emph{must be wrong}
\end{itemize}
\end{frame}

\begin{frame}
\begin{itemize}
\item cosmic microwave background
\item large-scale structure
\item Solar System
\end{itemize}
\end{frame}

\begin{frame}
\begin{itemize}
\item WIMPs
\item axions or other particle exotica
\item tiny black holes
\item \emph{not} dust, rocks, molecules, or anything atomic
\end{itemize}
\end{frame}

\begin{frame}
~[direct detection image]
\end{frame}

\begin{frame}
~[simulated dark-matter halo]
\end{frame}

\begin{frame}
~[streams around galaxies]
\end{frame}

\end{document}
