% Copyright 2013 David W. Hogg (NYU).  All rights reserved.

\documentclass{beamer}
\usepackage{amssymb,amsmath,mathrsfs}
\usepackage{listings}
\usecolortheme{default}

%%% color commands
\newcommand{\whiteonblack}{%
  \colorlet{fg}{white}
  \colorlet{bg}{black}
  \setbeamercolor{normal_text}{fg=white,bg=black}
  \setbeamercolor{background canvas}{fg=white,bg=black}
  \setbeamercolor{alerted_text}{fg=yellow}
  \setbeamercolor{example_text}{fg=white}
  \setbeamercolor{structure}{fg=white}
  \setbeamercolor{palette_quaternary}{fg=white}
}
\newcommand{\blackonwhite}{%
  \colorlet{fg}{black}
  \colorlet{bg}{white}
  \setbeamercolor{normal_text}{fg=black,bg=white}
  \setbeamercolor{background canvas}{fg=black,bg=white}
  \setbeamercolor{alerted_text}{fg=blue}
  \setbeamercolor{example_text}{fg=black}
  \setbeamercolor{structure}{fg=black}
  \setbeamercolor{palette_quaternary}{fg=black}
}
\xdefinecolor{pink}{rgb}{1.0,0.9,0.9}

%%% size and shape commands
\newlength{\figurewidth}
\setlength{\figurewidth}{0.9\textwidth}
\newlength{\figureheight}
\setlength{\figureheight}{0.9\textheight}

%%% text commands
\newcommand{\project}[1]{\textsl{#1}}
  \newcommand{\an}{\project{Astrometry.net}}
  \newcommand{\flickr}{\project{flickr}}
  \newcommand{\gaia}{\project{Gaia}}
  \newcommand{\galex}{\project{GALEX}}
  \newcommand{\GALEX}{\galex}
  \newcommand{\hst}{\project{HST}}
  \newcommand{\hipparcos}{\project{Hipparcos}}
  \newcommand{\lsst}{\project{LSST}}
  \newcommand{\sdss}{\project{SDSS}}
  \newcommand{\sdssiii}{\project{SDSS-III}}
  \newcommand{\boss}{\sdssiii\ \project{BOSS}}
  \newcommand{\osss}{\project{OSSS}}
  \newcommand{\vo}{\project{VO}}
  \newcommand{\rttd}{\project{Right Thing To Do}$^{\mbox{\scriptsize\sffamily{TM}}}$}
\newcommand{\foreign}[1]{\textit{#1}}
\newcommand{\latin}[1]{\foreign{#1}}
  \newcommand{\cf}{\latin{cf.}}
  \newcommand{\eg}{\latin{e.g.}}
  \newcommand{\etal}{\latin{et~al.}}
  \newcommand{\etc}{\latin{etc.}}
  \newcommand{\ie}{\latin{i.e.}}
  \newcommand{\vs}{\latin{vs.}}

%%% math-mode commands
\newcommand{\unit}[1]{\mathrm{#1}}
  \newcommand{\rad}{\unit{rad}}
  \newcommand{\s}{\unit{s}}
  \newcommand{\yr}{\unit{yr}}
  \newcommand{\km}{\unit{km}}
  \newcommand{\kmps}{\km\,\s^{-1}}
\newcommand{\mmatrix}[1]{\boldsymbol{#1}}
\newcommand{\tv}[1]{\boldsymbol{#1}}
\newcommand{\dd}{\mathrm{d}}
\newcommand{\given}{\,|\,}

%%% code commands
\lstset{language=Python,
        basicstyle=\tiny\ttfamily,
        showspaces=false,
        showstringspaces=false,
        tabsize=2,
        breaklines=false,
        breakatwhitespace=true,
        identifierstyle=\ttfamily,
        keywordstyle=\bfseries\color[rgb]{0.133,0.545,0.133},
        commentstyle=\color[rgb]{0.133,0.545,0.133},
        stringstyle=\color[rgb]{0.627,0.126,0.941},
        }
 % hogg standard colors

\begin{document}

\begin{frame}
\begin{itemize}
\item Universe is expanding
\item black holes
\item dark matter
\item dark energy
\item multiverse
\item how \emph{could} we know these things?
  \begin{itemize}
  \item \emph{simple} (compact) explanations of data
  \item explain \emph{multiple kinds} of data
  \item much more compact than the alternatives
  \end{itemize}
\end{itemize}
\end{frame}

\begin{frame}
~[galaxy cluster]
\end{frame}

\begin{frame}
~[Milky-Way-like galaxy]
\end{frame}

\begin{frame}
~[same, zoomed out]
\end{frame}

\begin{frame}
\begin{itemize}
\item cold
\item non-interacting (or weak-scale)
\item Gaussian
\item adiabatic
\item (cosmological constant)
\item this model \emph{must be wrong}
\end{itemize}
\end{frame}

\begin{frame}
\begin{itemize}
\item cosmic microwave background
\item large-scale structure
\item Solar System
\end{itemize}
\end{frame}

\begin{frame}
\begin{itemize}
\item WIMPs
\item axions or other particle exotica
\item tiny black holes
\item \emph{not} dust, rocks, molecules, or anything atomic
\end{itemize}
\end{frame}

\begin{frame}
~[direct detection image]
\end{frame}

\begin{frame}
~[simulated dark-matter halo]
\end{frame}

\begin{frame}
~[streams around galaxies]
\end{frame}

\end{document}
